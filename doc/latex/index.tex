This is a program which computes the Riemann sums and trapeziod rule sums of an expression over a given interval with a given number of steps. It is used for the estimation of the area under a curve, or the integral. For information on Riemann sums, see \href{http://en.wikipedia.org/wiki/Riemann_sum}{\tt http://en.wikipedia.org/wiki/Riemann\_\-sum} . The entry point is \hyperlink{ram_8c_a212f497d0e60a46a30582320c2fefef}{\_\-main()} in \hyperlink{ram_8c}{ram.c}. Various functions dealing with popping or examining elements on the expression stack are located in \hyperlink{jstack_8c}{jstack.c}. For a reentrant (besides temporarily enlarging the stack) function to call which does riemann sums or trapeziod rule calculations, see \hyperlink{ram_8c_549bc93c3536bde8ffbcca9bb5035fda}{riemann\_\-sum()} or \hyperlink{ram_8c_0415ba140e87ce3dca1acc1017f36fc7}{trap\_\-rule()} in \hyperlink{ram_8c}{ram.c}. \hypertarget{index_Arguments}{}\section{Arguments}\label{index_Arguments}
Note: The arguments are taken from the expression stack on the calculator. \hypertarget{index_expression}{}\subsection{expression}\label{index_expression}
The expression to be integrated. \hypertarget{index_var}{}\subsection{var}\label{index_var}
The variable of integration. \hypertarget{index_lower_bounds}{}\subsection{lower bounds}\label{index_lower_bounds}
The lower boundary of the interval. \hypertarget{index_upper_bounds}{}\subsection{upper bounds}\label{index_upper_bounds}
The upper boundary of the interval. \hypertarget{index_number_of_steps}{}\subsection{number of steps}\label{index_number_of_steps}
The number of sections to divide the region into; the number of sampling points. Increasing this will increase the accuracy, but will make the program take longer to execute. \hypertarget{index_function_index}{}\subsection{function index}\label{index_function_index}
This determines which computation will be performed according to the following: \par
 0 = left riemann sum \par
 1 = middle riemann sum \par
 2 = right riemann sum \par
 3 = trapezoid rule 