\hypertarget{ram_8c}{
\section{ram.c File Reference}
\label{ram_8c}\index{ram.c@{ram.c}}
}
{\tt \#include \char`\"{}ram.h\char`\"{}}\par
\subsection*{Functions}
\begin{CompactItemize}
\item 
void \hyperlink{ram_8c_a212f497d0e60a46a30582320c2fefef}{\_\-main} ()
\item 
float \hyperlink{ram_8c_f129554ebce2d89e788eb46f3a3547bc}{value\_\-at\_\-point} (ESI expression, ESI var, float x)
\item 
float \hyperlink{ram_8c_549bc93c3536bde8ffbcca9bb5035fda}{riemann\_\-sum} (ESI expression, ESI var, float lower\_\-bounds, float upper\_\-bounds, unsigned int numsteps, float offset)
\item 
float \hyperlink{ram_8c_0415ba140e87ce3dca1acc1017f36fc7}{trap\_\-rule} (ESI expression, ESI var, float lower\_\-bounds, float upper\_\-bounds, unsigned int numsteps)
\end{CompactItemize}


\subsection{Function Documentation}
\hypertarget{ram_8c_a212f497d0e60a46a30582320c2fefef}{
\index{ram.c@{ram.c}!\_\-main@{\_\-main}}
\index{\_\-main@{\_\-main}!ram.c@{ram.c}}
\subsubsection{\setlength{\rightskip}{0pt plus 5cm}void \_\-main ()}}
\label{ram_8c_a212f497d0e60a46a30582320c2fefef}


The entry point of the program. Takes arguments as follow from the expression stack: an expression to be integrated, the variable of integration, the lower bounds (float), the upper bounds (float), number of steps, and the function index, which instructs the \hyperlink{ram_8c_a212f497d0e60a46a30582320c2fefef}{\_\-main()} to call the corresponding function. \par
 0=lram \par
 1=mram \par
 2=rram \par
 3=trapezoid rule \hypertarget{ram_8c_549bc93c3536bde8ffbcca9bb5035fda}{
\index{ram.c@{ram.c}!riemann\_\-sum@{riemann\_\-sum}}
\index{riemann\_\-sum@{riemann\_\-sum}!ram.c@{ram.c}}
\subsubsection{\setlength{\rightskip}{0pt plus 5cm}float riemann\_\-sum (ESI {\em expression}, \/  ESI {\em var}, \/  float {\em lower\_\-bounds}, \/  float {\em upper\_\-bounds}, \/  unsigned int {\em numsteps}, \/  float {\em offset})}}
\label{ram_8c_549bc93c3536bde8ffbcca9bb5035fda}


Returns the rieman sum, for expression from lower\_\-bounds to upper\_\-bounds, with numsteps number of steps, using var as the variable of integration. Assuming lram is the default operation, offset $\ast$ dx (with var as x), would be added to the x coordinate of every sample point. \hypertarget{ram_8c_0415ba140e87ce3dca1acc1017f36fc7}{
\index{ram.c@{ram.c}!trap\_\-rule@{trap\_\-rule}}
\index{trap\_\-rule@{trap\_\-rule}!ram.c@{ram.c}}
\subsubsection{\setlength{\rightskip}{0pt plus 5cm}float trap\_\-rule (ESI {\em expression}, \/  ESI {\em var}, \/  float {\em lower\_\-bounds}, \/  float {\em upper\_\-bounds}, \/  unsigned int {\em numsteps})}}
\label{ram_8c_0415ba140e87ce3dca1acc1017f36fc7}


This function is very similar to rram, except that it computes the trapezoid rule instead, and has no argument offset, because there are no variations on the trapeziod rule. \hypertarget{ram_8c_f129554ebce2d89e788eb46f3a3547bc}{
\index{ram.c@{ram.c}!value\_\-at\_\-point@{value\_\-at\_\-point}}
\index{value\_\-at\_\-point@{value\_\-at\_\-point}!ram.c@{ram.c}}
\subsubsection{\setlength{\rightskip}{0pt plus 5cm}float value\_\-at\_\-point (ESI {\em expression}, \/  ESI {\em var}, \/  float {\em x})}}
\label{ram_8c_f129554ebce2d89e788eb46f3a3547bc}


Returns the float value of expression, when var is set to the value x. 